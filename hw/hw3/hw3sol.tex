\documentclass{article}
\usepackage{amsmath}
\usepackage{amssymb}
\usepackage{enumitem}
\usepackage{graphicx}
\usepackage[margin=1in]{geometry}
\usepackage[overload]{empheq}
\usepackage{subcaption}
\usepackage{listings}
\usepackage{color}
\usepackage{polynom}

% These two lines are from this StackExchange post: https://tex.stackexchange.com/a/177270
\usepackage{sectsty}
\allsectionsfont{\mdseries}

% The following lines, up to \title, are from this StackExchange post: http://tex.stackexchange.com/questions/14071/ddg#14072
\makeatletter
\renewcommand*\env@matrix[1][\arraystretch]{%
  \edef\arraystretch{#1}%
  \hskip -\arraycolsep
  \let\@ifnextchar\new@ifnextchar
  \array{*\c@MaxMatrixCols c}}
\makeatother

\title{Homework 3 Solutions}
\author{Math 198: Math for Machine Learning}
\date{}

\begin{document}
\maketitle

\noindent
Due Date:  \\
Name: \\
Student ID:

\section{Practice with Determinant and Trace}
\begin{enumerate}
\item Let $T: \mathbb{R}^2 \to \mathbb{R}^2$ be a linear map that sends $(1,0) \mapsto (2,0)$ and $(0,1) \mapsto (3,4)$. What is $\det(T)$? \\
{\color{blue} The matrix representation of $T$ with respect to the standard basis is $$\mathbf{T} = \begin{bmatrix} 2 & 3 \\ 0 & 4 \end{bmatrix}$$ Therefore, $\det(T) = \det(\mathbf{T}) = 2*4 - 3*0 = 8$.}
\item What is $\text{tr}(T)$? \\
{\color{blue} $\text{tr}(T) = \text{tr}(\mathbf{T}) = 2 + 4 = 6$.}
\item Let $U$ be a proper subspace of a vector space $V$. Let $\mathbf{P}$ be a map onto $U$. What is $\det(\mathbf{P})$? \\
{\color{blue} Because $U$ is a proper subspace of $V$, it has lower dimension than $V$. Therefore, $\mathbf{P}$ is not invertible, so its determinant is 0.}
\end{enumerate}

\section{Proofs about Determinant and Trace}
\noindent Let $\mathbf{A}$ be an arbitrary square matrix.
\begin{enumerate}
\item Prove that, if $\mathbf{A}$ is invertible, then $\text{det}(\mathbf{A}^{-1}) = \text{det}(\mathbf{A})^{-1}$. \\
{\color{blue} Recall that $\det(\mathbf{AB}) = \det(\mathbf{A})\det(\mathbf{B})$. We have $1 = \det(\mathbf{I}) = \det(\mathbf{AA}^{-1}) = \det(\mathbf{A})\det(\mathbf{A}^{-1})$. Therefore, $\det(\mathbf{A}^{-1}) = \frac{1}{\det(\mathbf{A})}$.}
\item Conclude that if $\text{det}(\mathbf{A}) = 0$, $\mathbf{A}$ is not invertible. \\
{\color{blue} Suppose toward a contradiction that $\mathbf{A}$ is invertible and $\det(\mathbf{A}) = 0$. Then $\det(\mathbf{A}^{-1}) = \frac{1}{0}$, so the determinant for $\mathbf{A}^{-1}$ is undefined. But the determinant is defined for all matrices. So, if $\det(\mathbf{A}) = 0$, then $\mathbf{A}$ cannot be invertible.}
\item Let $\mathbf{B}$ be an invertible matrix. Prove $\text{tr}(\mathbf{A}) = \text{tr}(\mathbf{BAB}^{-1})$. \\
{\color{blue} Since trace is invariant under cyclic permutations, $\text{tr}(\mathbf{BAB}^{-1}) = \text{tr}(\mathbf{B}^{-1}\mathbf{BA}) = \text{tr}(\mathbf{A})$.}
\end{enumerate}


\section{Computing Eigenvalues and Eigenvectors}
Let $$\mathbf{A} = \begin{bmatrix}[1.5] 4 & 1 & -1 \\ \frac{3}{2} & \frac{7}{2} & -\frac{3}{2} \\ \frac{1}{2} & \frac{-1}{2} & \frac{5}{2} \end{bmatrix}$$
\begin{enumerate}
\item Find $p_{\mathbf{A}}(\lambda)$, the characteristic polynomial of $\mathbf{A}$. \\
{\color{blue} \begin{align*}
p_{\mathbf{A}}(\lambda) &= 
\det(\mathbf{A} - \lambda\mathbf{I}) \\
&= \det\begin{bmatrix}[1.5] 4 - \lambda & 1 & -1 \\ \frac{3}{2} & \frac{7}{2} - \lambda & -\frac{3}{2} \\ \frac{1}{2} & -\frac{1}{2} & \frac{5}{2} - \lambda \end{bmatrix} \\
&= (4 - \lambda)\det\begin{bmatrix}[1.5] \frac{7}{2} - \lambda & -\frac{3}{2} \\ -\frac{1}{2} & \frac{5}{2} - \lambda\end{bmatrix} - \det\begin{bmatrix}[1.5] \frac{3}{2} & -\frac{3}{2} \\ \frac{1}{2} & \frac{5}{2} - \lambda\end{bmatrix} - \det\begin{bmatrix}[1.5] \frac{3}{2} & \frac{7}{2} - \lambda \\ \frac{1}{2} & -\frac{1}{2}\end{bmatrix} \\
&= (4 - \lambda)\big(\big(\frac{7}{2} - \lambda\big)\big(\frac{5}{2} - \lambda\big) - \frac{3}{4}\big) - \big(\frac{3}{2}\big(\frac{5}{2} - \lambda\big) + \frac{3}{4}\big) + \big(\frac{3}{4} + \frac{1}{2}\big(\frac{7}{2} - \lambda\big)\big) \\
&= (4 - \lambda)(\lambda^2 - 6\lambda + 8) - \big(\frac{9}{2} - \frac{3\lambda}{2}\big) + \big(\frac{5}{2} - \frac{\lambda}{2}\big) \\
&= -\lambda^3 + 10\lambda^2 - 31\lambda + 30
\end{align*}}
\item Using $p_{\mathbf{A}}(\lambda)$, compute the eigenvalues of $\mathbf{A}$. \\
{\color{blue} Observe that $30 = 2 * 3 * 5$. Therefore, to factor $p_{\mathbf{A}}(\lambda)$, we can start by seeing if any of these values are roots. Using polynomial long division (omitted), we find $$\frac{-\lambda^3 + 10\lambda^2 - 31\lambda + 30}{\lambda - 2} = -\lambda^2 + 8\lambda - 15 = -(\lambda - 3)(\lambda - 5)$$ So, we have $p_{\mathbf{A}}(\lambda) = -(\lambda - 2)(\lambda - 3)(\lambda - 5)$. Therefore, the eigenvalues of $\mathbf{A}$ are 2, 3, and 5.}
\item Find the eigenvectors of $\mathbf{A}$. \\
{\color{blue} We first find the eigenvector corresponding to $\lambda_1 = 5$: \begin{align*}
\mathbf{v}_1 &\in \ker(\mathbf{A} - 5\mathbf{I}) \\
&\in \ker\begin{bmatrix}[1.5] -1 & 1 & -1 \\ \frac{3}{2} & -\frac{3}{2} & -\frac{3}{2} \\ \frac{1}{2} & -\frac{1}{2} & -\frac{5}{2}\end{bmatrix}
\end{align*} Observe that $(\mathbf{A} - 5\mathbf{I})_1 = -(\mathbf{A} - 5\mathbf{I})_2$. Therefore, $$\begin{bmatrix}[1.5] -1 & 1 & -1 \\ \frac{3}{2} & -\frac{3}{2} & -\frac{3}{2} \\ \frac{1}{2} & -\frac{1}{2} & -\frac{5}{2}\end{bmatrix}\begin{bmatrix}1 \\ 1 \\ 0\end{bmatrix} = \mathbf{0}$$ so $[1\ 1\ 0]^{\top}$ is an eigenvector of $\mathbf{A}$ corresponding to $\lambda_1 = 5$. We turn next to $\lambda_2 = 3$:  \begin{align*}
\mathbf{v}_1 &\in \ker(\mathbf{A} - 3\mathbf{I}) \\
&\in \ker\begin{bmatrix}[1.5] 1 & 1 & -1 \\ \frac{3}{2} & \frac{1}{2} & -\frac{3}{2} \\ \frac{1}{2} & -\frac{1}{2} & -\frac{1}{2}\end{bmatrix}
\end{align*} Observe that $(\mathbf{A} - 3\mathbf{I})_1 = -(\mathbf{A} - 3\mathbf{I})_3$. Therefore, $$\begin{bmatrix}[1.5] 1 & 1 & -1 \\ \frac{3}{2} & \frac{1}{2} & -\frac{3}{2} \\ \frac{1}{2} & -\frac{1}{2} & -\frac{1}{2}\end{bmatrix}\begin{bmatrix}1 \\ 0 \\ 1\end{bmatrix} = \mathbf{0}$$ so $[1\ 0\ 1]^{\top}$ is an eigenvector of $\mathbf{A}$ corresponding to $\lambda_2 = 3$. Finally, for $\lambda_3 = 2$:  \begin{align*}
\mathbf{v}_1 &\in \ker(\mathbf{A} - 2\mathbf{I}) \\
&\in \ker\begin{bmatrix}[1.5] 2 & 1 & -1 \\ \frac{3}{2} & \frac{3}{2} & -\frac{3}{2} \\ \frac{1}{2} & -\frac{1}{2} & \frac{1}{2}\end{bmatrix}
\end{align*} Observe that $(\mathbf{A} - 2\mathbf{I})_2 = -(\mathbf{A} - 2\mathbf{I})_3$. Therefore, $$\begin{bmatrix}[1.5] 2 & 1 & -1 \\ \frac{3}{2} & \frac{3}{2} & -\frac{3}{2} \\ \frac{1}{2} & -\frac{1}{2} & \frac{1}{2}\end{bmatrix}\begin{bmatrix}0 \\ 1 \\ 1\end{bmatrix} = \mathbf{0}$$ so $[0\ 1\ 1]^{\top}$ is an eigenvector of $\mathbf{A}$ corresponding to $\lambda_3 = 2$.}
\end{enumerate}

\section{Proofs about Eigenvalues}
\noindent Let $\mathbf{A} \in \mathbb{R}^{n \times n}$ be a square matrix such that the eigenvectors of $\mathbf{A}$ are a basis for $\mathbb{R}^n$. Additionally, let $\lambda_i$, $1 \leq i \leq n$ be the eigenvalues of $\mathbf{A}$.
\begin{enumerate}
\item Prove that $\text{det}(\mathbf{A}) = \prod\limits_{i = 1}^n \lambda_i$. \\
{\color{blue} Let $\mathbf{Q\Lambda Q}^{-1}$ be the eigendecomposition of $\mathbf{A}$. Then $\det(\mathbf{A}) = \det(\mathbf{Q\Lambda Q}^{-1}) = \det(\mathbf{Q)\det(\Lambda)\det(Q}^{-1}) = \frac{\det(\mathbf{Q})}{\det(\mathbf{Q})}\det(\mathbf{\Lambda}) = \det(\mathbf{\Lambda})$. But $\mathbf{\Lambda}$ is diagonal, so $\det(\mathbf{\Lambda}) = \prod\limits_{i = 1}^n \mathbf{\Lambda}_{ii}$. The diagonal elements of $\mathbf{\Lambda}$ are the eigenvalues of $\mathbf{A}$, so therefore $\det(\mathbf{A}) = \prod\limits_{i=1}^n \lambda_i$.}
\item Prove that $\text{tr}(\mathbf{A}) = \sum\limits_{i = 1}^n \lambda_i$. \\
{\color{blue} Recall that trace is invariant under similarity. Since $\mathbf{A} \sim \mathbf{\Lambda}$, $\text{tr}(\mathbf{A}) = \text{tr}(\mathbf{\Lambda}) = \sum\limits_{i = 1}^n \mathbf{\Lambda}_{ii} = \sum\limits_{i = 1}^n \lambda_i$.}
\end{enumerate}

\end{document}
